% Options for packages loaded elsewhere
\PassOptionsToPackage{unicode}{hyperref}
\PassOptionsToPackage{hyphens}{url}
\PassOptionsToPackage{dvipsnames,svgnames,x11names}{xcolor}
%
\documentclass[
]{report}

\usepackage{amsmath,amssymb}
\usepackage{iftex}
\ifPDFTeX
  \usepackage[T1]{fontenc}
  \usepackage[utf8]{inputenc}
  \usepackage{textcomp} % provide euro and other symbols
\else % if luatex or xetex
  \usepackage{unicode-math}
  \defaultfontfeatures{Scale=MatchLowercase}
  \defaultfontfeatures[\rmfamily]{Ligatures=TeX,Scale=1}
\fi
\usepackage{lmodern}
\ifPDFTeX\else  
    % xetex/luatex font selection
\fi
% Use upquote if available, for straight quotes in verbatim environments
\IfFileExists{upquote.sty}{\usepackage{upquote}}{}
\IfFileExists{microtype.sty}{% use microtype if available
  \usepackage[]{microtype}
  \UseMicrotypeSet[protrusion]{basicmath} % disable protrusion for tt fonts
}{}
\makeatletter
\@ifundefined{KOMAClassName}{% if non-KOMA class
  \IfFileExists{parskip.sty}{%
    \usepackage{parskip}
  }{% else
    \setlength{\parindent}{0pt}
    \setlength{\parskip}{6pt plus 2pt minus 1pt}}
}{% if KOMA class
  \KOMAoptions{parskip=half}}
\makeatother
\usepackage{xcolor}
\setlength{\emergencystretch}{3em} % prevent overfull lines
\setcounter{secnumdepth}{-\maxdimen} % remove section numbering
% Make \paragraph and \subparagraph free-standing
\ifx\paragraph\undefined\else
  \let\oldparagraph\paragraph
  \renewcommand{\paragraph}[1]{\oldparagraph{#1}\mbox{}}
\fi
\ifx\subparagraph\undefined\else
  \let\oldsubparagraph\subparagraph
  \renewcommand{\subparagraph}[1]{\oldsubparagraph{#1}\mbox{}}
\fi


\providecommand{\tightlist}{%
  \setlength{\itemsep}{0pt}\setlength{\parskip}{0pt}}\usepackage{longtable,booktabs,array}
\usepackage{calc} % for calculating minipage widths
% Correct order of tables after \paragraph or \subparagraph
\usepackage{etoolbox}
\makeatletter
\patchcmd\longtable{\par}{\if@noskipsec\mbox{}\fi\par}{}{}
\makeatother
% Allow footnotes in longtable head/foot
\IfFileExists{footnotehyper.sty}{\usepackage{footnotehyper}}{\usepackage{footnote}}
\makesavenoteenv{longtable}
\usepackage{graphicx}
\makeatletter
\def\maxwidth{\ifdim\Gin@nat@width>\linewidth\linewidth\else\Gin@nat@width\fi}
\def\maxheight{\ifdim\Gin@nat@height>\textheight\textheight\else\Gin@nat@height\fi}
\makeatother
% Scale images if necessary, so that they will not overflow the page
% margins by default, and it is still possible to overwrite the defaults
% using explicit options in \includegraphics[width, height, ...]{}
\setkeys{Gin}{width=\maxwidth,height=\maxheight,keepaspectratio}
% Set default figure placement to htbp
\makeatletter
\def\fps@figure{htbp}
\makeatother

\usepackage{booktabs}
\usepackage{amsthm}
\usepackage{placeins}
\makeatletter
\def\thm@space@setup{%
  \thm@preskip=8pt plus 2pt minus 4pt
  \thm@postskip=\thm@preskip
}
\makeatother
\usepackage{adjustbox}
\usepackage{awesomebox}
\usepackage{color}
\usepackage{framed}
\setlength{\fboxsep}{.8em}
\usepackage[most]{tcolorbox}
\usepackage{blindtext}
\usepackage{amsmath}
\usepackage{amssymb}
\usepackage{bm}
\usepackage[finnish]{babel}
\usepackage{graphicx}
\usepackage{placeins}
\usepackage{overpic}
\usepackage{lmodern}
\usepackage{epsfig}
\usepackage{placeins}
\usepackage{xstring}     % Used for \IfEqCase


\definecolor{myboxcolor}{named}{blue} % Default box color

\definecolor{my-purple}{RGB}{204,180,225}
\newtcolorbox{defblock}[1]{%
    breakable,
    enhanced,
    coltext=black,
    colback=my-purple,      % Box color is used here
    colframe=myboxcolor!25!my-purple,     % Box color is used here
    detach title,
    after upper={\par\hfill\tcbtitle}        % Box title is used here 
}

\definecolor{my-orange}{RGB}{255,205,138}
\newtcolorbox{eblock}[1]{%
    breakable,
    enhanced,
    coltext=black,
    colback=my-orange,      % Box color is used here
    colframe=myboxcolor!25!my-orange,     % Box color is used here
    detach title,
    after upper={\par\hfill\tcbtitle}        % Box title is used here 
}


\newcommand{\Rspace}{\mathcal{R}}

\newcommand{\N}{\mathsf{N}}
\newcommand{\Cov}{\mathsf{Cov}}

\newcommand{\Prob}{\mathsf{P}}

\newcommand{\X}{\textbf{X}} 
\newcommand{\Y}{\textbf{Y}} 
\newcommand{\x}{\textbf{x}}                                   
\newcommand{\y}{\textbf{y}}  
\newcommand{\boldc}{\textbf{c}} 
\newcommand{\boldd}{\textbf{d}}  
\newcommand{\bolda}{\textbf{a}}  
\newcommand{\THETA}{\mx{\theta}}
\newcommand{\PHI}{\mx{\phi}}                                   
\newcommand{\VAREPSILON}{\mx{\varepsilon}}                                   
\newcommand{\hatVAREPSILON}{\mx{\hat{\VAREPSILON}}}
\newcommand{\boldP}{\textbf{P}}
\newcommand{\boldM}{\textbf{M}}
\newcommand{\z}{\mx{z}}
\newcommand{\A}{\textbf{A}}
\newcommand{\C}{\textbf{C}}
\newcommand{\hatMU}{\mx{\hat{\mu}}}
\newcommand{\SIGMA}{\mx{\Sigma}}
\newcommand{\ZERO}{\mx{0}}
\newcommand{\ONE}{\mx{1}}
\newcommand{\diag}{\textbf{I}}

\newcommand{\bl}[1]{\textcolor{blue}{#1}}
\newcommand{\rd}[1]{\textcolor{red}{#1}}
\newcommand{\gr}[1]{\textcolor{darkgreenx}{#1}}


%\newcommand\indep{\protect\mathpalette{\protect\independenT}{\perp}}
%\def\independenT#1#2{\mathrel{\rlap{$#1#2$}\mkern2mu{#1#2}}}
\makeatletter
\makeatother
\makeatletter
\makeatother
\makeatletter
\@ifpackageloaded{caption}{}{\usepackage{caption}}
\AtBeginDocument{%
\ifdefined\contentsname
  \renewcommand*\contentsname{Table of contents}
\else
  \newcommand\contentsname{Table of contents}
\fi
\ifdefined\listfigurename
  \renewcommand*\listfigurename{List of Figures}
\else
  \newcommand\listfigurename{List of Figures}
\fi
\ifdefined\listtablename
  \renewcommand*\listtablename{List of Tables}
\else
  \newcommand\listtablename{List of Tables}
\fi
\ifdefined\figurename
  \renewcommand*\figurename{Figure}
\else
  \newcommand\figurename{Figure}
\fi
\ifdefined\tablename
  \renewcommand*\tablename{Table}
\else
  \newcommand\tablename{Table}
\fi
}
\@ifpackageloaded{float}{}{\usepackage{float}}
\floatstyle{ruled}
\@ifundefined{c@chapter}{\newfloat{codelisting}{h}{lop}}{\newfloat{codelisting}{h}{lop}[chapter]}
\floatname{codelisting}{Listing}
\newcommand*\listoflistings{\listof{codelisting}{List of Listings}}
\makeatother
\makeatletter
\@ifpackageloaded{caption}{}{\usepackage{caption}}
\@ifpackageloaded{subcaption}{}{\usepackage{subcaption}}
\makeatother
\makeatletter
\@ifpackageloaded{tcolorbox}{}{\usepackage[skins,breakable]{tcolorbox}}
\makeatother
\makeatletter
\@ifundefined{shadecolor}{\definecolor{shadecolor}{rgb}{.97, .97, .97}}
\makeatother
\makeatletter
\makeatother
\makeatletter
\makeatother
\ifLuaTeX
  \usepackage{selnolig}  % disable illegal ligatures
\fi
\IfFileExists{bookmark.sty}{\usepackage{bookmark}}{\usepackage{hyperref}}
\IfFileExists{xurl.sty}{\usepackage{xurl}}{} % add URL line breaks if available
\urlstyle{same} % disable monospaced font for URLs
\hypersetup{
  pdftitle={Luku 2 - Tieteellinen tieto, tilastot ja arkitieto yhteiskunnassa},
  colorlinks=true,
  linkcolor={blue},
  filecolor={Maroon},
  citecolor={Blue},
  urlcolor={Blue},
  pdfcreator={LaTeX via pandoc}}

\title{Luku 2 - Tieteellinen tieto, tilastot ja arkitieto
yhteiskunnassa}
\usepackage{etoolbox}
\makeatletter
\providecommand{\subtitle}[1]{% add subtitle to \maketitle
  \apptocmd{\@title}{\par {\large #1 \par}}{}{}
}
\makeatother
\subtitle{Tiivistelmä}
\author{}
\date{}

\begin{document}
\maketitle
\ifdefined\Shaded\renewenvironment{Shaded}{\begin{tcolorbox}[enhanced, boxrule=0pt, frame hidden, interior hidden, breakable, sharp corners, borderline west={3pt}{0pt}{shadecolor}]}{\end{tcolorbox}}\fi

\hypertarget{luvun-ydinviesti}{%
\section{Luvun ydinviesti}\label{luvun-ydinviesti}}

Luku käsittelee tieteen yhteiskunnallista merkitystä ja sitä, mitä tiede
on ja miten sitä tehdään.

Tilastotiede näyttelee keskeistä roolia lähes kaikessa tieteellisessä
tutkimuksessa, sillä se tarjoaa yleisenä menetelmätieteenä keinoja
erilaisten tutkimuskysymysten ja niihin liitettävien hypoteesien
testaamiseen.

\textbf{Tämän luvun ytimessä on siis tilastotieteen rooli uuden tutkitun
tiedon tuottamisessa!}

\hypertarget{mituxe4-on-tiede}{%
\section{Mitä on tiede?}\label{mituxe4-on-tiede}}

Tiede on tiedonhankintaa ympäröivästä maailmasta ja sen ilmiöistä.

Tietoa voidaan hankkia monin eri tavoin mutta milloin siitä tulee
tiedettä?

Tieteessä uuden tiedon hankintaa ohjaa tieteellinen menetelmä, joka
varmistaa hankitun tiedon oikeellisuuden ja luotettavuuden niin hyvin
kuin se suinkin on mahdollista!

Tiede on (ainakin):

\begin{itemize}
\item
  \textbf{Järjestelmällistä:} tieteellinen tiedonhankinta on
  yhteiskunnallisesti organisoitu tutkimusinstituutioiden tehtäväksi.
\item
  \textbf{Järkiperäistä:} tieteellinen tiedonhankinta on kumulatiivinen
  prosessi, jossa tiedonhankinnan menetelmät ovat muokkantuneet
  tarkoituksenmukaisiksi, luotettaviksi ja järkeviksi.
\end{itemize}

\hypertarget{keskeiset-termit}{%
\section{Keskeiset termit}\label{keskeiset-termit}}

Seuraavaksi käsitellään luvun 2 keskeiset termit. Nämä kannattaa
sisäistää hyvin, ei pelkästään tämän kurssin läpipääsyn vaan myös
loppuopintojen sujuvuuden takia. Alasta riippumatta!

\hypertarget{tieteellinen-teoria}{%
\section{Tieteellinen teoria}\label{tieteellinen-teoria}}

\begin{defblock}{}

\textbf{Tieteellinen teoria}

\begin{itemize}
\item
  Tieteelliset teoriat ovat hyvin perusteltuja kuvauksia ja selityksiä,
  ympäröivän maailmamme toiminnasta tai siellä esiintyvien ilmiöiden
  välisistä yhteyksistä.

  \begin{itemize}
  \tightlist
  \item
    Ne ovat luotetuin, täsmällisin ja kattavin tieteellisen tiedon
    muoto.
  \end{itemize}
\item
  Tieteellisen teorian pyrkimys on selittää ja ennustaa sen kohteena
  olevaa ilmiötä.

  \begin{itemize}
  \tightlist
  \item
    Se on luonteeltaan induktiivinen ja alisteinen muutoksille tai jopa
    hylkäämiselle (\emph{falsifikaatiolle}) empiirisen todistusaineiston
    niin osoittaessa.
  \end{itemize}
\end{itemize}

\end{defblock}

\hypertarget{hypoteesi}{%
\section{Hypoteesi}\label{hypoteesi}}

\begin{defblock}{}

\textbf{Hypoteesi}

\begin{itemize}
\item
  Hypoteesi tarkoittaa teorioista johdettua tai aikaisemman tutkimuksen
  perusteella esitettyä ennakoitua ratkaisua tai selitystä tutkittavaan
  ongelmaan.
\item
  Hypoteesi ilmaistaan teoriaa koskevana väitteenä, jonka
  paikkansapitävyyttä halutaan tutkia.
\item
  \textbf{Nollahypoteesiksi} asetetaan usein jokin tyypillinen ja
  odotettavissa oleva tulos, esimerkiksi ettei kahden mitatun ilmiön
  välillä ole yhteyttä. Nollahypoteesia \emph{ei todisteta tai
  hyväksytä}!
\item
  Sopivalla nollahypoteesin valinnalla voidaan testata väitteitä, jotka
  ovat teorian ennustusten kanssa yhteneviä tai siitä poikkeavia.
\end{itemize}

\end{defblock}

\hypertarget{tieteellinen-menetelmuxe4}{%
\section{Tieteellinen menetelmä}\label{tieteellinen-menetelmuxe4}}

\begin{defblock}{}

\textbf{Tieteellinen menetelmä}

\begin{itemize}
\item
  Tieteenalan vallitseva sekä yleisesti hyväksytty ja hyväksi todettu
  tapa hankkia tietoa, joka on mm.

  \begin{itemize}
  \item
    \textbf{Objektiivinen ja looginen}: tutkimuskohde tutkijan
    mielipiteistä riippumaton, tieto syntyy näiden vuorovaikutuksen
    tuloksena.
  \item
    \textbf{Kriittinen:} asetettavien ja testattavien hypoteesien tulee
    olla avoimia tieteelliselle kritiikille, mm. julkisen tarkastamisen
    kautta.
  \item
    \textbf{Autonominen:} tieteellisen tutkimuksen tuloksia arvioidaan
    tiedeyhteisön sisällä, ei niiden ulkopuolisen vaikuttavuuden
    kontekstissa.
  \item
    \textbf{Edistyvä:} tieteelliset teoriat kehittyvät yhä tarkemmiksi
    kuvauksiksi tai selityksiksi.
  \item
    \textbf{Toistettava:} tieteellinen tutkimus voidaan toistaa tai
    uusintaa.
  \end{itemize}
\end{itemize}

\end{defblock}

\hypertarget{mituxe4-on-tutkimus}{%
\section{Mitä on tutkimus?}\label{mituxe4-on-tutkimus}}

Tutkimus on kumulatiivinen ja itseään korjaava prosessi, jossa uutta
tietoa hankitaan tieteellistä menetelmää käyttämällä.

Ympäröivää maailmaa kuvaava teoria on aina yksinkertaistus ja
tutkimustuloksiin vaikuttaa menetelmän ohella tutkittavan ilmiön
luontainen satunnaisuus.

Uusi tieto on siis aina epävarmaa ja tarvitsee vahvistusta, joka tekee
teorioiden kehityksestä hidasta.

\begin{itemize}
\item
  Mikäli tutkimuksen tulokset ovat linjassa teorian tekemien ennustusten
  kanssa, teoria vahvistuu eli se ``\emph{verifioidaan''.}
\item
  Jos taas tutkimuksen tulokset poikkeavat teorian tekemistä
  ennustuksista, ne tulkitaan teorian empiiriseksi vastaväitteeksi eli
  ``\emph{falsifikaatioksi}''.
\end{itemize}



\end{document}
