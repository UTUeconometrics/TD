% Options for packages loaded elsewhere
\PassOptionsToPackage{unicode}{hyperref}
\PassOptionsToPackage{hyphens}{url}
\PassOptionsToPackage{dvipsnames,svgnames,x11names}{xcolor}
%
\documentclass[
]{report}

\usepackage{amsmath,amssymb}
\usepackage{iftex}
\ifPDFTeX
  \usepackage[T1]{fontenc}
  \usepackage[utf8]{inputenc}
  \usepackage{textcomp} % provide euro and other symbols
\else % if luatex or xetex
  \usepackage{unicode-math}
  \defaultfontfeatures{Scale=MatchLowercase}
  \defaultfontfeatures[\rmfamily]{Ligatures=TeX,Scale=1}
\fi
\usepackage{lmodern}
\ifPDFTeX\else  
    % xetex/luatex font selection
\fi
% Use upquote if available, for straight quotes in verbatim environments
\IfFileExists{upquote.sty}{\usepackage{upquote}}{}
\IfFileExists{microtype.sty}{% use microtype if available
  \usepackage[]{microtype}
  \UseMicrotypeSet[protrusion]{basicmath} % disable protrusion for tt fonts
}{}
\makeatletter
\@ifundefined{KOMAClassName}{% if non-KOMA class
  \IfFileExists{parskip.sty}{%
    \usepackage{parskip}
  }{% else
    \setlength{\parindent}{0pt}
    \setlength{\parskip}{6pt plus 2pt minus 1pt}}
}{% if KOMA class
  \KOMAoptions{parskip=half}}
\makeatother
\usepackage{xcolor}
\setlength{\emergencystretch}{3em} % prevent overfull lines
\setcounter{secnumdepth}{-\maxdimen} % remove section numbering
% Make \paragraph and \subparagraph free-standing
\ifx\paragraph\undefined\else
  \let\oldparagraph\paragraph
  \renewcommand{\paragraph}[1]{\oldparagraph{#1}\mbox{}}
\fi
\ifx\subparagraph\undefined\else
  \let\oldsubparagraph\subparagraph
  \renewcommand{\subparagraph}[1]{\oldsubparagraph{#1}\mbox{}}
\fi


\providecommand{\tightlist}{%
  \setlength{\itemsep}{0pt}\setlength{\parskip}{0pt}}\usepackage{longtable,booktabs,array}
\usepackage{calc} % for calculating minipage widths
% Correct order of tables after \paragraph or \subparagraph
\usepackage{etoolbox}
\makeatletter
\patchcmd\longtable{\par}{\if@noskipsec\mbox{}\fi\par}{}{}
\makeatother
% Allow footnotes in longtable head/foot
\IfFileExists{footnotehyper.sty}{\usepackage{footnotehyper}}{\usepackage{footnote}}
\makesavenoteenv{longtable}
\usepackage{graphicx}
\makeatletter
\def\maxwidth{\ifdim\Gin@nat@width>\linewidth\linewidth\else\Gin@nat@width\fi}
\def\maxheight{\ifdim\Gin@nat@height>\textheight\textheight\else\Gin@nat@height\fi}
\makeatother
% Scale images if necessary, so that they will not overflow the page
% margins by default, and it is still possible to overwrite the defaults
% using explicit options in \includegraphics[width, height, ...]{}
\setkeys{Gin}{width=\maxwidth,height=\maxheight,keepaspectratio}
% Set default figure placement to htbp
\makeatletter
\def\fps@figure{htbp}
\makeatother

\usepackage{booktabs}
\usepackage{amsthm}
\usepackage{placeins}
\makeatletter
\def\thm@space@setup{%
  \thm@preskip=8pt plus 2pt minus 4pt
  \thm@postskip=\thm@preskip
}
\makeatother
\usepackage{adjustbox}
\usepackage{awesomebox}
\usepackage{color}
\usepackage{framed}
\setlength{\fboxsep}{.8em}
\usepackage[most]{tcolorbox}
\usepackage{blindtext}
\usepackage{amsmath}
\usepackage{amssymb}
\usepackage{bm}
\usepackage[finnish]{babel}
\usepackage{graphicx}
\usepackage{placeins}
\usepackage{overpic}
\usepackage{lmodern}
\usepackage{epsfig}
\usepackage{placeins}
\usepackage{xstring}     % Used for \IfEqCase


\definecolor{myboxcolor}{named}{blue} % Default box color

\definecolor{my-purple}{RGB}{204,180,225}
\newtcolorbox{defblock}[1]{%
    breakable,
    enhanced,
    coltext=black,
    colback=my-purple,      % Box color is used here
    colframe=myboxcolor!25!my-purple,     % Box color is used here
    detach title,
    after upper={\par\hfill\tcbtitle}        % Box title is used here 
}

\definecolor{my-orange}{RGB}{255,205,138}
\newtcolorbox{eblock}[1]{%
    breakable,
    enhanced,
    coltext=black,
    colback=my-orange,      % Box color is used here
    colframe=myboxcolor!25!my-orange,     % Box color is used here
    detach title,
    after upper={\par\hfill\tcbtitle}        % Box title is used here 
}


\newcommand{\Rspace}{\mathcal{R}}

\newcommand{\N}{\mathsf{N}}
\newcommand{\Cov}{\mathsf{Cov}}

\newcommand{\Prob}{\mathsf{P}}

\newcommand{\X}{\textbf{X}} 
\newcommand{\Y}{\textbf{Y}} 
\newcommand{\x}{\textbf{x}}                                   
\newcommand{\y}{\textbf{y}}  
\newcommand{\boldc}{\textbf{c}} 
\newcommand{\boldd}{\textbf{d}}  
\newcommand{\bolda}{\textbf{a}}  
\newcommand{\THETA}{\mx{\theta}}
\newcommand{\PHI}{\mx{\phi}}                                   
\newcommand{\VAREPSILON}{\mx{\varepsilon}}                                   
\newcommand{\hatVAREPSILON}{\mx{\hat{\VAREPSILON}}}
\newcommand{\boldP}{\textbf{P}}
\newcommand{\boldM}{\textbf{M}}
\newcommand{\z}{\mx{z}}
\newcommand{\A}{\textbf{A}}
\newcommand{\C}{\textbf{C}}
\newcommand{\hatMU}{\mx{\hat{\mu}}}
\newcommand{\SIGMA}{\mx{\Sigma}}
\newcommand{\ZERO}{\mx{0}}
\newcommand{\ONE}{\mx{1}}
\newcommand{\diag}{\textbf{I}}

\newcommand{\bl}[1]{\textcolor{blue}{#1}}
\newcommand{\rd}[1]{\textcolor{red}{#1}}
\newcommand{\gr}[1]{\textcolor{darkgreenx}{#1}}


%\newcommand\indep{\protect\mathpalette{\protect\independenT}{\perp}}
%\def\independenT#1#2{\mathrel{\rlap{$#1#2$}\mkern2mu{#1#2}}}
\makeatletter
\makeatother
\makeatletter
\makeatother
\makeatletter
\@ifpackageloaded{caption}{}{\usepackage{caption}}
\AtBeginDocument{%
\ifdefined\contentsname
  \renewcommand*\contentsname{Table of contents}
\else
  \newcommand\contentsname{Table of contents}
\fi
\ifdefined\listfigurename
  \renewcommand*\listfigurename{List of Figures}
\else
  \newcommand\listfigurename{List of Figures}
\fi
\ifdefined\listtablename
  \renewcommand*\listtablename{List of Tables}
\else
  \newcommand\listtablename{List of Tables}
\fi
\ifdefined\figurename
  \renewcommand*\figurename{Figure}
\else
  \newcommand\figurename{Figure}
\fi
\ifdefined\tablename
  \renewcommand*\tablename{Table}
\else
  \newcommand\tablename{Table}
\fi
}
\@ifpackageloaded{float}{}{\usepackage{float}}
\floatstyle{ruled}
\@ifundefined{c@chapter}{\newfloat{codelisting}{h}{lop}}{\newfloat{codelisting}{h}{lop}[chapter]}
\floatname{codelisting}{Listing}
\newcommand*\listoflistings{\listof{codelisting}{List of Listings}}
\makeatother
\makeatletter
\@ifpackageloaded{caption}{}{\usepackage{caption}}
\@ifpackageloaded{subcaption}{}{\usepackage{subcaption}}
\makeatother
\makeatletter
\@ifpackageloaded{tcolorbox}{}{\usepackage[skins,breakable]{tcolorbox}}
\makeatother
\makeatletter
\@ifundefined{shadecolor}{\definecolor{shadecolor}{rgb}{.97, .97, .97}}
\makeatother
\makeatletter
\makeatother
\makeatletter
\makeatother
\ifLuaTeX
  \usepackage{selnolig}  % disable illegal ligatures
\fi
\IfFileExists{bookmark.sty}{\usepackage{bookmark}}{\usepackage{hyperref}}
\IfFileExists{xurl.sty}{\usepackage{xurl}}{} % add URL line breaks if available
\urlstyle{same} % disable monospaced font for URLs
\hypersetup{
  pdftitle={Luku 3 - Tilastotiede tieteenalana},
  pdfauthor={Roope Rihtamo},
  colorlinks=true,
  linkcolor={blue},
  filecolor={Maroon},
  citecolor={Blue},
  urlcolor={Blue},
  pdfcreator={LaTeX via pandoc}}

\title{Luku 3 - Tilastotiede tieteenalana}
\usepackage{etoolbox}
\makeatletter
\providecommand{\subtitle}[1]{% add subtitle to \maketitle
  \apptocmd{\@title}{\par {\large #1 \par}}{}{}
}
\makeatother
\subtitle{Tiivistelmä}
\author{Roope Rihtamo}
\date{}

\begin{document}
\maketitle
\ifdefined\Shaded\renewenvironment{Shaded}{\begin{tcolorbox}[borderline west={3pt}{0pt}{shadecolor}, interior hidden, boxrule=0pt, sharp corners, enhanced, breakable, frame hidden]}{\end{tcolorbox}}\fi

\hypertarget{luvun-ydinviesti}{%
\section{Luvun ydinviesti}\label{luvun-ydinviesti}}

Tilastotiede on useille korkeakouluopinnot aloittavalle varsin vieras
tieteenala. Tämä johtuu siitä että tilastotiedettä ei juuri opeteta
lukioissa tai ammattikouluissa. Ei ainakaan sellaisena alana, joksi se
korkeakouluissa mielletään!

Tässä luvussa tarkastellaankin siis \textbf{mitä tilastotiede oikeastaan
on?}

Erityisesti keskitymme erottelemaan tilastotieteen sen lähitieteistä,
matematiikasta ja tietojenkäsittelytieteestä. Syvempi tarkastelu
mahdollistaisi myös tilastotieteen ja tieteenfilosofian välisen suhteen
tarkastelun, mutta jätetään se tällä kertaa väliin!

\hypertarget{keskeiset-termit}{%
\section{Keskeiset termit}\label{keskeiset-termit}}

\begin{defblock}{}
\textbf{Populaatio}

Konkreettinen tai hypoteettinen tutkimuskohteiden joukko, joka koostuu
kaikista tilastoyksiköistä.

\end{defblock}

\begin{defblock}{}
\textbf{Tilastoyksikkö ja tilastollinen muuttuja}

Populaation muodostavilta tilastoyksiköiltä tarkastellaan tilastollisia
muuttujia, joita voidaan mitata tai havaita.

\end{defblock}

\begin{defblock}{}
\textbf{Havainto ja havaintoaineisto}

Havainto muodostuu tilastoyksikön tarkasteltavien tilastollisten
muuttujien havaitusta arvoista ja havaintoaineisto, data, on näiden
havaintojen joukko.

\end{defblock}

\hypertarget{tilastotieteen-karakterisointeja}{%
\section{Tilastotieteen
karakterisointeja}\label{tilastotieteen-karakterisointeja}}

Luotan tässä määritelmässä itseäni pätevämpiin.

Leo Törnqvist, Suomen ensimmäinen tilastotieteen professori:
``\emph{Tilastotiede on tietotuotannon teknologiaa}, \emph{jonka avulla
voidaan suorittaa kvantitatiivisten tietojen joukkotuotantoa ja
havaintoihin perustuvia tieteellisiä ja käytännöllisiä päätöksiä.}''

Ilkka Mellin (2004): ``\emph{Tilastotiede on yleinen menetelmätiede},
\emph{jota sovelletaan, jos reaalimaailman ilmiöstä halutaan tehdä
johtopäätöksiä ilmiötä kuvaavien kvantitatiivisten tai numeeristen
tietojen perusteella sellaisissa tilanteissa, joissa tietoihin liittyy
epävarmuutta tai satunnaisuutta.''}

Mark Twain: \href{https://fi.wikipedia.org/wiki/Mark_Twain}{``Vale,
emävale, tilasto''}, teoksessaan \emph{Chapters from My Autobiography}
vuonna 1907, mutta koska valtaosa ``modernin'' tilastotieteen
teoriakehityksestä on tapahtunut vasta Twainin teoksen julkaisun
jälkeen, niin tällä lentävällä lausahduksella ei ole mitään tekemistä
nykyisten tilastollisten menetelmien kanssa!

\hypertarget{mituxe4-tilastotiede-ei-ole}{%
\section{Mitä tilastotiede ei ole}\label{mituxe4-tilastotiede-ei-ole}}

Toisin kuin usein luullaan, tilastotiede ei ole \emph{ainoastaan}
tilastojen, eli \textbf{numeeristen tietojen järjestelmällisten
kokoelmien}, tuotantoa ja niiden harrastamista.

Tilastotiede sijoittuu tieteiden kentässä matematiikan, filosofian ja
tietojenkäsittelytieteen rinnalle. Tästä huolimatta se ei kuitenkaan ole
yksiselitteisesti minkään näiden osa-alue.

Tilastotiede ei ole matematiikkaa, sillä se lähestyy tieteellistä
ongelmanratkaisua eri tavoin: tilastotiede on aina konteksti- ja
aineistopohjaista ja perustuu induktiiviseen päättelyyn.

Tilastotiede ei ole tietojenkäsittelytieteen osa-alue, koska niillä on
omat erilliset ja ehjät teoriapohjansa joista jälkimmäinen ei perustu
ajatukselle satunnaisista reaalimaailman ilmiöistä.

\hypertarget{mituxe4-tilastotiede-ainakin-on}{%
\section{Mitä tilastotiede (ainakin)
on}\label{mituxe4-tilastotiede-ainakin-on}}

Tieteellistä tietoa hankitaan tieteellisellä menetelmällä (ks. luku 2),
jonka avulla tutkitaan mielenkiinnon kohteena olevaa ilmiötä tai sen
generoimaa kvantitatiivista mutta epävarmaa tietoa sisältävää aineistoa.

Tilastotieteessä kehitetään menetelmiä, jotka antavat eri alojen
tutkijoille yhtenevät ja yleisesti hyväksytyt raamit, jotka
mahdollistavat (keinot) päättelyn epävarmuuden vallitessa ja keinot
tämän epävarmuuden mittaamiseen.

\begin{itemize}
\tightlist
\item
  \emph{Tilastotiede kehittää matemaattisia malleja satunnaisilmiöitä
  kuvaavia kvantitatiivisia tietoja generoiville prosesseille!}
\end{itemize}

Tilastotieteen voidaan katsoa kuuluvan ns. menetelmätieteisiin, joissa
kehitetään menetelmiä, mutta jolla on myös oma sovelluksista vapaa
teorianmuodostuksensa.

\hypertarget{tilastotieteen-osa-alueet}{%
\section{Tilastotieteen osa-alueet}\label{tilastotieteen-osa-alueet}}

Tilastotiede kehittyi pitkään ns. ``ongelmasta menetelmään'', kun
yhteiskunnan eri aloilla tarvittiin keinoja tilastojen analysointiin.
Tämä johti siihen, että tilastotiede on jakautunut moniin osa-alueisiin.
Seuraava karkea kahtiajako voidaan kuitenkin tehdä

\begin{defblock}{}
\textbf{Soveltava tilastotiede}

on nimensä mukaisesti teoreettisen tilastotieteen kehittämien
menetelmien soveltamista jonkin tutkimusalan empiiriseen ongelmaan.

\end{defblock}

\begin{defblock}{}
\textbf{Teoreettinen tilastotiede} kehittää (tilasto)matemaattisia
malleja kuvaamaan satunnaisilmiöitä, jotka generoivat reaalimaailman
ilmiöitä kuvaavia numeerisia tietoja, joihin liittyy epävarmuutta ja
satunnaisuutta.

\end{defblock}

\hypertarget{tilastotieteen-kritiikkiuxe4}{%
\section{Tilastotieteen kritiikkiä}\label{tilastotieteen-kritiikkiuxe4}}

Tilastotiedettä, kuten mitä tahansa alaa, on myös kritisoitu vuosien
saatossa eri tavoin. \emph{Usein kritiikki on aiheetonta ja perustuu
väärinymmärryksiin.}

\begin{itemize}
\item
  \textbf{``Yleismaailmallinen kritiikki''} tiivistyy seuraavaan
  ``\emph{Jos on toinen jalka jäässä ja toinen kiehuvassa vedessä, niin
  tilastotieteilijän mielestä ihmisellä on tällöin keskimäärin sopivan
  lämmin}''.
\item
  \textbf{Kritiikki matemaattisuutta kohtaan:} tilastotieteen
  matemaattinen esitystapa on paikoin vaikeasti lähestyttävää.
\item
  \textbf{Kritiikki yksnikertaistuksia kohtaan:} tilastotiedettä on
  kritisoitu siitä että se ei kykene riittävällä tasolla huomioimaan
  reaalimaailman kompleksisuutta.
\item
  \textbf{Temppukokoelmakritiikki:} tilastotiedettä pidetään paikoin
  vain kokoelmana erilaisia menetelmiä tai ``temppuja''.
\end{itemize}

\hypertarget{tilastotieteen-sovellusaloja-ja-rajatieteituxe4}{%
\section{Tilastotieteen sovellusaloja ja
``rajatieteitä''}\label{tilastotieteen-sovellusaloja-ja-rajatieteituxe4}}

Yleisenä menetelmätieteenä tilastotiedettä sovelletaan useilla eri
tieteenaloilla, joilla on oma erillinen teoriapohjansa sekä empiiriset
käytänteet ja täten substanssitietous on sovellettaessa tärkeää!

Jokaisella tieteenalalla, jonka tutkimusaineistot voidaan esittää
numeerisessa tai kvantitatiivisessa muodossa voi soveltaa/voisi
soveltaa/pitäisi soveltaa tilastollisia menetelmiä sekä
tutkimusaineistoja kerättäessä että niitä analysoitaessa.

Muutamia esimerkkejä aloista, joilla tilastotieteen soveltaminen on
muodostunut omaksi tutkimuskohteekseen: psykologia
(\href{https://en.wikipedia.org/wiki/Psychometrics}{psykometriikka}),
bio- ja lääketiede
(\href{https://en.wikipedia.org/wiki/Biometrics}{biometria}),
taloustiede
(\href{https://en.wikipedia.org/wiki/Econometrics}{ekonometria}) ja
kemia (\href{https://en.wikipedia.org/wiki/Chemometrics}{kemometria}).



\end{document}
